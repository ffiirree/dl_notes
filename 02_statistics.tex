\chapter{Statistics}

\section{Frequentist statistics VS Bayesian statistics}
\begin{quotation}
Throughout our subsequent discussions, we viewed $\theta$ as an unknown parameter of the world. 
This view of the $\theta$ as being constant-valued but unknown is taken in frequentist statistics.
In the frequentist this view of the world, $\theta$ is not random—it just happens to be unknown—and 
it's our job to come up with statistical procedures (such as maximum likelihood) to try to estimate 
this parameter. 

An alternative way to approach our parameter estimation problems is to take 
the Bayesian view of the world, and think of $\theta$ as being a random variable whose value is 
unknown. In this approach, we would specify a prior distribution $p(\theta)$ on $\theta$ that 
expresses our "prior beliefs" about the parameters. Given a training set $S = {(x_i, y_i)}$, 
make a prediction on a new value of x, we can then compute the posterior distribution on 
the parameters.(CS229)
\end{quotation}

\section{期望}
\begin{itemize}
    \item 若干个随机变量之和的期望等于各变量的期望之和,$E(X_1 + X_2 + ... + X_n) = E(X_1) + E(X_2) + ... + E(X_n)$
    \item 若干个独立随机变量之和的期望等于各变量的期望之和,$E(X_1 X_2 ... X_n) = E(X_1)E(X_2)...E(X_n)$
\end{itemize}
\section{方差}
均匀分布$X \sim R(a, b)$
其期望为
\begin{equation}
    \begin{split}
        E(X) &= \int_{a}^{b}x f(x) \mathrm{d}x		\\
        &= \frac{1}{b-a}\int_{a}^{b}x\mathrm{d}x	\\
        &= \frac{1}{b-a} \frac{1}{2} (b^2 - a^2)	\\
        &= \frac{1}{2}(b + a)
    \end{split}
\end{equation}
其方差为
\begin{equation}
    \begin{split}
        Var(X) &= E(X - EX)^2  \\
        &= E(X)^2 - (EX)^2 \\
        &= \int_{a}^{b} x^2 f(x) \mathrm{d}x - (EX)^2 \\
        &= \frac{1}{3(b-a)} (b^3 - a^3) - \frac{1}{4}(b+a)^2 \\
        &= \frac{1}{12}(b-a)^2
    \end{split}
\end{equation}
\\
对于正态分布$N(\mu, \sigma ^2)$,其期望$E(X) = \mu$,方差为$Var(X) = \sigma^2$.


\section{最小方差无偏估计}
