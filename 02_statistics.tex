\chapter{Statistics}
\begin{itemize}
    \item \textbf{sample space} The set, $\mathcal{S}$, of all possible outcomes of a particular experiment is called the sample space for the experiment.
    \item \textbf{event} An event is any collection of possible outcomes of an experiment, that is, any subset of $\mathcal{S}$.
    \item \textbf{conditional probability} If $A$ and $B$ are events in $\mathcal{S}$, and $P(B) > 0$, then the conditional probability of $A$ given $B$, written $P(A|B)$, is
        \begin{equation}
            P(A|B) = \frac{P(AB)}{P(B)}
        \end{equation}
    \item \textbf{Bayes' Rule} Let $A_1, A_2,...$ be a partition of the sample space, and let $B$ be any set. Then, for each $i = 1, 2, 3, ..$,
        \begin{equation}
            P(A_i|B) = \frac{P(B|A_i)P(A_i)}{\sum P(B|A_i)P(A_i)}
        \end{equation}
    \item \textbf{statistically independent} Two events, $A$ and $B$, are statistically independent if
        \begin{equation}
            P(AB) = P(A)P(B)
        \end{equation}
    \item \textbf{random variable} A random variable is a \textbf{function} from a sample space $\mathcal{S}$ into the real numbers.
    \item \textbf{identically distributed} The random variables $X$ and $Y$ are identically distributed if, for every set $A \in \mathcal{B}^1, P(X \in A) = P(Y \in A)$.
    \item \textbf{cdf} cumulative distribution function.
    \item \textbf{pmf} probability mass function.
    \item \textbf{pdf} probability density function.
    \item \textbf{moment/central moment} For each integer $n$, the $n$th moment of $X$, $\mu_n'$, is
        \begin{equation}
            \mu_n' = \E X^n
        \end{equation}
        The $n$th central moment of $X, \mu_n$, is
        \begin{equation}
            \mu_n = \E(X - \mu)^n
        \end{equation}
        where $\mu = \mu_1' = \E X$.
    \item \textbf{variance} The variance of a random variable $X$ is its second central moment, $Var X = \E(X - \E X)^2$. The positive square root of $Var X$ is the standard deviation of $X$.
    \item \textbf{$n$-dimensional random vector} An $n$-dimensional random vector is a function form a sample space $\mathcal{S}$ into $\mathfrak{R}^n$, $n$-dimensional Euclidean space.
    \item \textbf{joint pdf / joint pmf}
    \item \textbf{conditional pmf}
\end{itemize}

\section{Frequentist statistics VS Bayesian statistics}
\begin{quotation}
Throughout our subsequent discussions, we viewed $\theta$ as an unknown parameter of the world.
This view of the $\theta$ as being constant-valued but unknown is taken in frequentist statistics.
In the frequentist this view of the world, $\theta$ is not random—it just happens to be unknown—and
it's our job to come up with statistical procedures (such as maximum likelihood) to try to estimate
this parameter.

An alternative way to approach our parameter estimation problems is to take
the Bayesian view of the world, and think of $\theta$ as being a random variable whose value is
unknown. In this approach, we would specify a prior distribution $p(\theta)$ on $\theta$ that
expresses our "prior beliefs" about the parameters. Given a training set $S = {(x_i, y_i)}$,
make a prediction on a new value of x, we can then compute the posterior distribution on
the parameters.(CS229)
\end{quotation}

\section{Mean}
\begin{itemize}
    \item 若干个随机变量之和的期望等于各变量的期望之和,$\E(X_1 + X_2 + ... + X_n) = \E(X_1) + E(X_2) + ... + \E(X_n)$
    \item 若干个独立随机变量之和的期望等于各变量的期望之和,$\E(X_1 X_2 ... X_n) = \E(X_1)\E(X_2)...\E(X_n)$
\end{itemize}

\section{Variance}
均匀分布$X \sim R(a, b)$
其期望为
\begin{equation}
    \begin{split}
        \E(X) &= \int_{a}^{b}x f(x) \d x		\\
        &= \frac{1}{b-a}\in t_{a}^{b}x \d x	\\
        &= \frac{1}{b-a} \frac{1}{2} (b^2 - a^2)	\\
        &= \frac{1}{2}(b + a)
    \end{split}
\end{equation}
其方差为
\begin{equation}
    \begin{split}
        Var(X) &= \E(X - \E X)^2  \\
        &= \E(X)^2 - (\E X)^2 \\
        &= \int_{a}^{b} x^2 f(x) \d x - (\E X)^2 \\
        &= \frac{1}{3(b-a)} (b^3 - a^3) - \frac{1}{4}(b+a)^2 \\
        &= \frac{1}{12}(b-a)^2
    \end{split}
\end{equation}
\\
对于正态分布$\N(\mu, \sigma ^2)$,其期望$\E(X) = \mu$,方差为$Var(X) = \sigma^2$.


\section{Best Unbiased Estimators}


\section{Divergence}
In statistics and information geometry, \textbf{divergence} or a \textbf{contrast function} is a function
which establishes the "distance" of one probability distribution to the other on a statistical manifold.
The divergence is a weaker notion than that of the distance, \textit{in particular the divergence need not be symmetric},
and need not satisfy the triangle inequality. [\href{https://en.wikipedia.org/wiki/Divergence_(statistics)}{Wikipedia: Divergence (statistics)}]

Suppose $\mathcal{S}$ is a space of all probability distributions with common support.
Then a divergence on $\mathcal{S}$ is a function $D(\cdot\|\cdot): \mathcal{S} \times \mathcal{S} \rightarrow R$ satisfying
\begin{equation}
    \begin{split}
        D(p\|q) &\geq 0 \text{ for all }p, q \in S, \\
        D(p\|q) &= 0\text{ if and only if }p = q,
    \end{split}
\end{equation}

The two most important divergences are the \textbf{KL divergence} (relative entropy), which is central to information theory
and statistics, and the \textbf{squared Euclidean distance} (SED).

The two most important classes of divergences are the \textbf{$f$-divergences} and \textbf{Bregman divergences};
The only divergence that is both an $f$-divergence and a Bregman divergence is the KL divergence;
the SED is a Bregman divergence, but not an $f$-divergence.

\subsection{$f$-divergence}
An \textbf{$f$-divergence} is a function $D_f(P\|Q)$ that measures the difference between two probability
distributions $P$ and $Q$. It helps the intuition to think of the divergence as an average, weighted by the
function $f$, of the odds ratio given by $P$ and $Q$. [\href{https://en.wikipedia.org/wiki/F-divergence}{Wikipedia: $f$-divergence}]

\subsubsection{Definition}
Let $P$ and $Q$ be two probability over a space $\Omega$ such that $P$ is absolutely continuous with respect to
$Q$. Then, for a convex function $f$ such that $f(1)=0$, the $f$-divergence of $P$ from $Q$ is defined as
\begin{equation}
    D_f(P\|Q) = \int_{\Omega}f\left(\frac{\d P}{\d Q}\right) \d Q
\end{equation}
If $P$ and $Q$ are both absolutely continuous with respect to a reference distribution $\mu$ on $\Omega$ then their
probability densities $p$ and $q$ satisfy $\d P = p \d \mu$ and $\d Q = q \d \mu$. In this case the $f$-divergence can
be written as
\begin{equation}
    D_f(P\|Q) = \int_{\Omega} f\left(\frac{p(x)}{q(x)}\right) q(x) \d \mu(x)
\end{equation}

\subsubsection{Instances of $f$-divergences}

\begin{center}
    \begin{tabular}{cc}
        \toprule
        Divergence & $f(\cdot)$ \\
        \midrule
        KL-divergence & $t\log{t}$ \\
        reverse KL-Divergence & $-\log{t}$ \\
        Jensen-Shannon Divergence & $(t+1)\log(\frac{2}{t+1}) + t\log t$ \\
        squared Hellinger distance & $(\sqrt{t} - 1)^2, 2(1 - \sqrt{t})$ \\
        Total variation distance & $\frac{1}{2}|t-1|$ \\
        Pearson $\mathcal{X}^2$-divergence & $(t-1)^2, t^2 - 1, t^2 - t$ \\
        Neyman $\mathcal{X}^2$-divergence(reverse Pearson) & $\frac{1}{t} - 1, \frac{1}{t} - t$ \\
        \bottomrule
    \end{tabular}
\end{center}

\subsubsection{Properties}
\begin{itemize}
    \item \textbf{Non-negativity} : the $f$-divergence is always positive; it's zero if and only if
    the measures $P$ and $Q$ coincide. This follows immediately from Jensen's inequlity:
    \begin{equation}
        D_f(P\|Q) = \int f \left(\frac{\d P}{\d Q}\right) \d Q \geq f\left(\frac{\d P}{\d Q} \d Q\right) = f(1) = 0
    \end{equation}
    \item \textbf{Monotonicity}
    \item \textbf{Joint Convexity}
\end{itemize}

\subsection{Kullback–Leibler divergence / Relative entropy}
KLD  is a measure of how one probability distribution is different from a second, reference probability distribution.

Consider two probability distribution $P$ and $Q$. Usually, $P$ reresents the data, the observations, or a probability
distribution precisely measured. $Q$ represnts instead a theory, a model, a description or an approximation of $P$.
The KL divergence is then interperted as the average difference of the number of bits required for encoding samples of $P$
using a code optimized for $Q$ rather than one optimized for $P$

\subsubsection{Definition}
\begin{equation}
    \begin{split}
        D_{KL}(P\|Q)
        &= \E_{x \sim P(x)}[\log \frac{P(x)}{Q(x)}] \\
        &= \E_{x \sim P(x)} [\log P(x) - \log Q(x)]
    \end{split}
\end{equation}

\section{Entropy}
\subsection{Informational value}
The basic idea of information theory is that the "informational value" of a communicated message depends on the
degree to which the content of the message is surprising. If an event is very probable, it is no surprise
when that event happens as expected; hence transmission of such a message carries very little new information.
However, if an event is unlikely to occur, it is much more informative to learn that the event happened or will happen.

The information content (also called the surprisal) of an event $E$ is a function which decreases the probability $p(E)$
of an event increases, defined by $I(E) = - \log p(E)$.

\subsection{Entropy}

In information theory, the entropy of a random variable is the average level of "information", "surprise", or "uncertainty" inherent in the variable's possible outcomes.
\begin{equation}
    H(x) = -\E[\log P(x)]
\end{equation}

\subsection{Cross Entropy}
The cross entropy between two probability distributions measures the average number of bits needed to identify an
event from a set of possibilities, if a coding scheme is used based on a given probability distribution $Q$, rather
than the "true" distribution $P$. The cross entropy for two distributions $P$ and $Q$ over the same probability
space is thus defined as follows
\begin{equation}
    H(P, Q) = \E_{x \sim P(x)}[- \log Q(x)] = H(P(x)) + KL(P(x)\|Q(x))
\end{equation}
When $H(P)$ is constant, minimizing the $H(P, Q)$ is equivalent to minimizing the KL divergence $KL(P\|Q)$.

\subsection{Conditional Entropy}
In information theory, the conditional entropy quantifies the amount of information needed to describe the outcome of a random variable
$Y$ given that the value of another random variable $X$ is known.
\begin{equation}
    H(Y|X) = -\sum P(x, y) \log \frac{P(x, y)}{P(x)}
\end{equation}

\section{Mutual Information}
In probability theory and information theory, the mutual information (MI) of two random variables is a
measure of the \textbf{mutual dependence between the two variables}. More specifically, it quantifies the "amount of information"
obtained about one random variable through observing the other random variable. The concept of mutual information is
intimately linked to that of entropy of a random variable, a fundamental notion in information theory that quantifies
the expected "amount of information" held in a random variable.
\begin{equation}
    \begin{split}
        I(X;Y)
        &= KL(P(X,Y)\|P(X)P(Y)) \\
        &= \E_x[KL(P(Y|X)\|P(Y))] \\
        &= \E_y[KL(P(X|Y)\|P(X))] \\
        &= H(X) - H(X|Y) \\
        &= H(Y) - H(Y|X) \\
        &= H(X) + H(Y) - H(X, Y) \\
        &= H(X, Y) - H(X|Y) - H(Y|X) \\
        &= \sum_y\sum_x P(x, y) \log \frac{P(x, y)}{P(x)P(y)}
    \end{split}
\end{equation}


\section{Maximum likelihood estimation}
\textbf{Frequentist statistics}

深度学习来就是用模型$Q(x;\theta)$来估计数据的真实分布$P(x)$,对于一组确定的数据集$X$,在样本已被观察到的情况下,需要找到使得$Q(X; \theta)$出现可能性最大的一组参数$\theta$,也就是
最大似然估计:
\begin{equation}
    \begin{split}
        \hat{\theta}_{MLE}
        &= \arg \max_\theta Q(X; \theta) \\
        &= \arg \max_{\theta} \prod _{i=1}^m Q(x^i; \theta) \\
        &\Rightarrow \arg \max_{\theta} \sum_{i=0}^m \log Q(x^i; \theta) \\
        &\Rightarrow \arg \min_{\theta} - \sum_{i=0}^m \log Q(x^i; \theta) \\
        &\Rightarrow \arg \min_{\theta} - \E _{x \sim P(x)}[\log Q(x; \theta)]
    \end{split}
\end{equation}

\subsection{MLE \& KLD}
如果使用KLD来衡量模型$Q(x;\theta)$和真实分布$P(x)$之间的差距,其中$\E_{x \sim P(x)}\log P(x)$为常量,则有
\begin{equation}
    \begin{split}
    \arg \min KL(P(x)\|Q(x;\theta))
        &= \arg \min_{\theta} \E_{x \sim P(x)}[\log P(x) - \log Q(x;\theta)] \\
        &\Rightarrow \arg \min_{\theta} - \E_{x \sim P(x)}[\log Q(x;\theta)]
    \end{split}
\end{equation}

在估计真实分布的情况下,计算KLD和使用MLE是等价的。

\section{When P(x) is constant}
When P(x) is contant, these equtions are equivalent.
\begin{itemize}
    \item $ KLD = KL(P(x)\|Q(x;\theta)) = \E_{x \sim P(x)}[\log P(x) - \log Q(x; \theta)] = - \E_{x \sim P(x)}[\log Q(x;\theta)] $
    \item $ CE = H(P, Q) = \E_{x \sim P(x)}[- \log Q(x)] = KL(P(x)\|Q(x)) = - \E_{x \sim P(x)}[\log Q(x;\theta)] $
    \item $ MLE = \arg \min - \E _{x \sim P(x)}[\log Q(x; \theta)]$
\end{itemize}


\section{Maximum A Posteriori}
$P(\theta)$先验概率分布,$P(X|\theta)$是似然函数,根据贝叶斯定理,可用以下公式计算后验概率
\begin{equation}
    P(\theta|X) = \frac{P(X|\theta)P(\theta)}{P(X)}
\end{equation}
模型估计时,估计整个后验概率分布$P(\theta|X)$,如需给出一个模型,通常取后验概率最大的模型。
\begin{equation}
    \begin{split}
        \hat{\theta}_{MAP} &= \arg \max P(\theta | X) \\
        &= \arg \min -\log P(\theta | X) \\
        &= \arg \min -\log P(X|\theta) - \log P(\theta) + \log P(X) \\
        &= \arg \min -\log P(X|\theta) - \log P(\theta)
    \end{split}
\end{equation}
$\log P(X)$与$\theta$无关,可以丢掉。$-\log P(X|\theta)$其实就是NLL,所以MLE和MAP不同在$- \log P(\theta)$。
假定先验是一个高斯分布
\begin{equation}
    P(\theta) = C \times e^{-\frac{\theta^2}{2\sigma^2}}
\end{equation}
那么
\begin{equation}
    \log P(\theta) = C + \frac{\theta^2}{2\sigma^2}
\end{equation}

在MAP中,使用一个高斯分布的先验等价于在MLE中采用$L^2$的正则化。MAP贝叶斯推断提供了一个直观的方法来设计复杂但可解释的
正则化,更复杂的惩罚项可以通过混合高斯分布作为先验得到,而不是一个单独的高斯分布。


预测新的观察数据$x$时,计算数据对后验概率分布的期望值:
\begin{equation}
    p(x|X) = \int p(x | \theta, X)p(\theta|X)\mathrm{d}\theta
\end{equation}
