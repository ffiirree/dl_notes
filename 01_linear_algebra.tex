\chapter{Linear Algebra}

\section{Eigenvalues and Eigenvectors}
\begin{equation}
    Au = \lambda u
\end{equation}
特征值$\lambda$代表线性变化的伸缩倍数,特征向量$u$代表变换的方向

\section{奇异值分解}
\textbf{定义} 对于$A \in C^{m \times n}$,$rank(A = r$,矩阵$A^HA$的特征值为
$\lambda_1 \geqslant \lambda_2 \geqslant ... \geqslant \lambda_r > 0$,
$\lambda_{r+1} = \lambda_{r+2} = ... = \lambda_{n} = 0$,称正数$\sigma_i = \sqrt{\lambda_i}(i = 1,2,...,r) $
为矩阵$A$的\textbf{奇异值}
\\
\begin{equation}
    \begin{split}
        A
        &= U \Sigma V^H \\
        &= (u_1\ u_2\ \cdot \cdot \cdot\ u_m)
        \begin{pmatrix}
            \lambda_1 &  &
        \end{pmatrix}
        \begin{pmatrix}
            v_1^H \\
            v_2^H \\
            \cdot \\
            \cdot \\
            \cdot \\
            v_n^H
        \end{pmatrix}
        \\
        &= \sigma_1 \boldsymbol{u}_1 \boldsymbol{v}_1^H + \sigma_2 \boldsymbol{u}_2 \boldsymbol{v}_1^H + \cdot\cdot\cdot + \sigma_r \boldsymbol{u}_r \boldsymbol{v}_r^H
    \end{split}
\end{equation}

\subsection{降维和图像压缩}
将图片作为矩阵进行奇异值分解,提取前n个奇异值,则可以达到图像压缩的目的.
